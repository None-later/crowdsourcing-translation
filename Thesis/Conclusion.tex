\onehalfspaced

\section{Conclusion}
In this paper, we propose two mechanisms to optimize cost: the translation reducing method and the translator reducing method. They have different applicable scenarios for large corpus construction.
The translation reducing method works if there exists a specific requirement that the quality control must reach a certain threshold. 
This model is most effective when reasonable amounts of pre-existing professional translations are available for setting the models threshold. 
The translator reducing method is very simple and easy to implement. This approach is inspired by the intuition that workers' performance is consistent. The translator reducing method is suitable for crowdsourcing tasks which do not have specific requirements about the quality of the translations, 
or when only very limited amounts of gold standard data are available.
